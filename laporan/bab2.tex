\section{BAB 2}
\section{MATERI}
\subsection{2.1 Software Define Network}
\ \ \ \ Software Defined Network (SDN) adalah istilah yang merujuk pada konsep/paradigma baru dalam mendisain, mengelola dan
mengimplementasikan jaringan, terutama untuk mendukung kebutuhan dan inovasi di bidang ini yg semakin lama semakin
kompleks. Konsep dasar SDN adalah dengan melakukan pemisahan eksplisit antara control dan forwarding plane, serta
kemudian melakukan abstraksi sistem dan meng-isolasi kompleksitas yg ada pada komponen atau sub-sistem dengan
mendefinisikan antar-muka (interface) yg standard.

Beberapa aspek penting dari SDN adalah :
\liststyleLii
\begin{enumerate}
\item Adanya pemisahan secara fisik/eksplisit antara forwarding/data-plane dan control-plane
\item Antarmuka standard (vendor-agnosic) untuk memprogram perangkat jaringan
\item Control-plane yang terpusat (secara logika) atau adanya sistem operasi jaringan yang mampu membentuk peta logika
(logical map) dari seluruh jaringan dan kemudian memrepresentasikannya melalui (sejenis) API (Application Programming
Interface)
\item Virtualisasi dimana beberapa sistem operasi jaringan dapat mengkontrol bagian-bagian (slices atau substrates) dari
perangkat yang sama.
\end{enumerate}

Dalam konsep SDN, tersedia open interface yg memungkinkan sebuah entitas software/aplikasi untuk mengendalikan
konektivitas yg disediakan oleh sejumlah sumber-daya jaringan, mengendalikan aliran trafik yg melewatinya serta
melakukan inspeksi terhadap atau memodifikasi trafik tersebut.

Arsitektur SDN dapat dilihat sebagai 3 lapis/bidang:
\liststyleLiii
\begin{enumerate}
\item Infrastruktur (data-plane / infrastructure layer)
\end{enumerate}
Terdiri dari elemen jaringan yg dapat mengatur SDN Datapath sesuai dengan instruksi yg diberikan melalui
Control-Data-Plane Interface (CDPI)

\liststyleLiii
\setcounter{saveenum}{\value{enumi}}
\begin{enumerate}
\setcounter{enumi}{\value{saveenum}}
\item Kontrol (control plane / layer)
\end{enumerate}
Entitas kontrol (SDN Controller) mentranslasikan kebutuhan aplikasi dengan infrastruktur dengan memberikan instruksi yg
sesuai untuk SDN Datapath serta memberikan informasi yg relevan dan dibutuhkan oleh SDN Application

\liststyleLiii
\setcounter{saveenum}{\value{enumi}}
\begin{enumerate}
\setcounter{enumi}{\value{saveenum}}
\item Aplikasi (application plane / layer)
\end{enumerate}
Berada pada lapis teratas, berkomunikasi dengan sistem via NorthBound Interface (NBI)


\subsection{2.2 Mininet}
\ \ \ \ Mininet merupakah salah satu emulator jaringan sdn yang bisa menjalankan sebuah jaringan virtual yang mana dalam
jaringan tersebut terdapat host, switch, controller serta link antar virtual device yang ada. Host pada mininet
menggunakan sebuah perangkat lunak jaringan linux. Switch mininet dapat mendukung protokol openflow \ dengan routing
yang fleksible serta compatible dengan software define network. 

Beberapa keuntungan dari penggunaan emulator mininet:
\liststyleLiv
\begin{enumerate}
\item Dapat melakukan emulasi jaringan SDN dengan jumlah yang besar.
\item Mengaktifkan beberapa mengembang untuk berkerja secara independen pada topologi yang sama.
\item Memungkinkan pengujian topologi yang kompleks tanpa menggunakan jaringan fisik.
\item Mendukung custom topologi yang mudah dimodifikasi.
\item Menyediakan API python yang mudah dan dapat diperluas untuk pembuatan dan eksperimen jaringan
\end{enumerate}

Beberapa kekurangan dari penggunaan emulator mininet:
\liststyleLv
\begin{enumerate}
\item Tidak memiliki GUI yang lebih interaktif dan ramah terhadap pemula.
\item Tidak mudah dalam memahaminya.
\item Pada node tidak jelas perangkat apa yang sedang digunakan.
\item Tidak adanya perangkat wirelles pada mininet.
\end{enumerate}


\subsection{2.3 Floodlight}
\ \ \ \ Floodlight adalah sebuah controller SDN yang cukup populer, controller ini merupakan kontribusi dari Big Big Switch
Networks ke komunitas open source. Floodlight adalah sebuah pengendali \ openflow yang dibangung menggunakan bahasa
pemrograman java yang menggunakan lisensi Apache non-OSGI. 

Arsitektur pada floodlight ini bersifat modular, maksudnya adalah komponen pada floodlight mudah dikembangkan sesuai
dengan keinginan. 

Komponen komponen pada floodlight ini meliputi
\liststyleLvi
\begin{enumerate}
\item Manajemen topologi.
\item Manajemen perangkat meliputi MAC Address dan IP Address.
\item Path computation.
\item Infrastruktu dengan akses web untuk management.
\item Counter store.
\item Menggunakan database untuk menyimpan datanya defaultnya menggunakan memory.
\item Menyediakan REST API untuk pengembangan notifikasi dan sejenisnya.
\end{enumerate}
