\section{BAB 1}
\section{PENDAHULUAN}
\subsection{1.1 Latar Belakang}
\ \ \ \ Pada saat ini perkembangan teknologi informasi berkembang sangat pesat, tidak terkecuali pada jaringan komputer.
Saat ini berkembang gagasan paradigma baru dalam mengelola jaringan komputer, yang disebut Software-Define Networking
(SDN). Software-Define Networking (SDN) adalah sebuah konsep pendekatan baru untuk mendesain, membangun dan mengelola
jaringan komputer dengan memisahkan control plane dan data plane. Konsep utama pada Software-Define Networking (SDN)
adalah sentralisasi kendali jaringan dengan semua pengaturan berada pada control plane.

Konsep SDN ini sangat memudahkan operator dan network administrator dalam mengelola jaringannya. SDN juga mampu
memberikan solusi untuk permasalahan-permasalahan jaringan yang ada sekarang ini, seperti sulitnya mengintegrasikan
teknologi baru karena masalah perbedaan platform perangkat keras, kinerja yang buruk karena ada beberapa operasi yang
berlebihan pada protokol layer dan sulitnya menyediakan layanan-layanan baru.

Konsep dari SDN sendiri dapat mempermudah dan mempercepat inovasi pada jaringan sehingga diharapkan muncul ide-ide
baru yang lebih baik dan dapat dengan cepat diimplementasikan pada real network environment. 

\subsection{1.2 Tujuan}
\ \ \ \ Tujuan ditulisnya makalah ini adalah untuk memahami software define network serta mampu membangun sebuah jaringan
sdn dengan menggunakan controller floodlight.

\subsection{1.3 Rumusan Masalah}
\liststyleLi
\begin{enumerate}
\item Memahami software define network (SDN).
\item Membangun sebuah jaringan SDN pada mininet dengan controller floodlight.
\end{enumerate}
